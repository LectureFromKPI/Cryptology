\section{Основные направления, аспекты защиты информаци}

\subsection{Цели, задачи ЗИ}
\begin{enumerate}
	\item Конфидициальность
	\item Целостность
	\item Аутентичность - подтверждение подлинности сообщения (подвид целостности)
	\item Доступность
	\item Наблюдаемость - отслеживание доступа
	\item Юридическая значимость
\end{enumerate}

\subsection{Направления и методы ЗИ}
\begin{enumerate}
	\item Правовые
	\item Нормативно-методические
	\item Организационные
	\item Непосредственные (физические)
	\item Технические - защита от утечки через каналы информации
	\begin{itemize}
		\item электромагнитная
		\item аккустический
		\item виброаккустический
		\item оптический
		\item криптографические
		\item стеганографический (стегано - крыша) - скрывается сам файт передачи сообщения
		\item методы квантовой криптографии (1983)
		\item морально-психологические
	\end{itemize}
\end{enumerate}

\subsection{Основные понятия криптологии}
Криптология делится на криптографию и криптоанализ

Открытый текст - сообщение, подлежащее шифрованию.
\begin{eqnarray}
	X = x_1, ... x_n
	M = m_1m_2...m_n
	x_i, m_j \in Z_r
\end{eqnarray}
$ Z_r $ - алфавит из $ r $ букв
\begin{equation}
	Z_r = \left\lbrace 0,1,...,r-1 \right\rbrace 
\end{equation}
Шифрованный текст (криптограмма) (ШТ) - ;


Зашифрование - процедура ОТ->ШТ с использованием открытых или секретных ключей
Расшифрование - процедура ШТ->ОТ законным пользователем с использованием секретных ключей
Секретный ключ - параметр, управляющий процессом шифрования.
\begin{equation}
	K = k_1k_2...k_s, \qquad k_i \in Z_q
\end{equation}
$ \keyspace $ - пространство ключей.
$ |\keyspace| $ - мощность

\subsection{Классификация криптографии}
\begin{enumerate}
	\item Классическая криптография (до начала 20 века) - шифры перестановки и шифры перестановки
	\item Механические шифровальные машины (конец 19-го - начало 20-го века)
	\item Электромеханические (пример. Энигма)
	\item Современные криптосистемы (вторая половина 20го века) - аппаратные, программные и аппаратно-программные
	\begin{enumerate}
		\item Симметричные (с закрытым ключем) - блочные, потоковые;
		\item Ассиметричные (с открытым ключем) - с 1976 года;
		\item Квантовые - с 1983 года.
	\end{enumerate}
\end{enumerate}

\section{Классическая криптография}
\subsection{Шифры перестановки}
\begin{tabular}{|c|c|}
	\hline 
	ОТ & $ X = x_1...x_rx_{r+1} ... x_{2r} $ \\ 
	\hline 
	ШТ & $ Y = y_1...y_ry_{r+1} ... y_{2r} $ \\ 
	\hline 
\end{tabular} 

Секретный ключ шифрования $ K = \left( \begin{array}{cccc}
1 & r & r & r \\ 
i_1 & i_2 & ... & i_r
\end{array} \right) $

Секретный ключ расшифрования: $ K^{-1} $
\begin{equation}
	|\keyspace | = r!
\end{equation}

Зашифрование:
\begin{equation}
y_{(l-1)} = x_{(l-1)r+k(i)}
y_i = X_{k(i)}, при l=1
\end{equation}
Расшифрование:
\begin{equation}
	x_{(l-1)r+i} = y_{(l-1)r + K^{-1}(i)} i=1...r, l=1,2
\end{equation}

Тут еще был пример.

Еще один пример \textbf{Исторические шифры-перестановки}
\begin{enumerate}
	\item Шифр Скитала, 5век до н.э. Спарта - наматывалась лента на скитал(барабан). Аристотель криптоанализил (на конусе)
	\item Шифр простой табличной перестановки - открытый текст записывался по строкам, а считывался столбцам. Параметр - размер таблицы. Сложность $ n\cdot m $
	\item Шифр усложненной табличной перестановки - еще перестановки столбцов и строк использовались. Сложность $ n!m! $
	\item Решетка Кардана. $ |\keyspace| = 4^9 = 2^18$ В общем случае, для $ 2n\cdot 2n $ - $ |\keyspace| = 4^{n^2} $
	\item Магический квадрат
	\begin{equation}
		\begin{array}{ccc}
		4 & 9 & 2 \\ 
		3 & 5 & 7 \\ 
		8 & 1 & 6
		\end{array} 
	\end{equation}
\end{enumerate}
Савчук Михаил Николаевич
\subsection{Литература}
...