\usepackage{ifdraft}

\usepackage{fontspec}
\usepackage{xltxtra} 
\usepackage{polyglossia}   %% загружает пакет многоязыковой вёрстки

\setdefaultlanguage{ukrainian}  %% устанавливает главный язык документа
\setotherlanguage{english} %% объявляет второй язык документа
\setotherlanguage{russian} %% третий язык документа - для библиографии
\defaultfontfeatures{Ligatures=TeX,Mapping=tex-text}

%% Виставлення шрифтів документу
\newfontfamily{\cyrillicfont}{Times New Roman}
\newfontfamily{\cyrillicfontsf}{Arial}
\newfontfamily{\cyrillicfonttt}[Scale=0.95]{Courier New}

\usepackage{extsizes}
\usepackage{titlesec}


\usepackage{setspace}%[nodisplayskipstretch]
	\onehalfspacing
	\frenchspacing
	
\usepackage{graphicx} 	% для вставки картинок


\usepackage{amsfonts,amsmath,amsthm} % математические дополнения от АМС
\usepackage{unicode-math}
\usepackage{indentfirst} % отделять первую строку раздела абзацным отступом тоже
\usepackage[usenames,dvipsnames]{color} % названия цветов
\usepackage{ulem} % подчеркивания
\usepackage{tocloft}
\usepackage{import}
\usepackage{lastpage} % знаходить номер останньої сторінки за \pageref*{LastPage}~с.
\usepackage{etoolbox}
\usepackage[title,titletoc]{appendix}
\usepackage{pdfpages}


%пакети для таблиць
\usepackage{array}
\usepackage{tabulary}
\usepackage{tabularx}
\usepackage{float} 		% пакет для [H]-положення фігур
\usepackage{rotating}	% пакет для перевертання блоків
\usepackage{makecell}
\usepackage{multirow} % улучшенное форматирование таблиц