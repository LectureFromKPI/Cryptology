$$ a\equiv b mod m 
\Longleftrightarrow 
	\begin{array}{c}
		a=km+r, \\
		b=tk+r 
	\end{array}
\Longleftrightarrow 
a-b : m $$

Сравнение. 
\subsection{Свойства сравнения}
Рассмотрим свойства сравнения:
\begin{eqnarray}
a\equiv b mod m\\
c\equiv d mod m
\end{eqnarray}
$ a+c \equiv b+d mod m $


$ ka \equiv kb mod m $ - обе части можно умножить на некоторое число.

\paragraph{Почленное перемножение}
\begin{eqnarray}
a\equiv b mod m\\
c\equiv d mod m
\end{eqnarray}
-> 
$ ac \equiv bd mod m $
\subparagraph{Доказательство}
\begin{eqnarray}
	ac = (km+r)(cm + r_1) = ()m+rr_1 \\
	bd = (tm+r)(hm + r_1) = ()m+rr_1
\end{eqnarray}

Из этого можно получить, что:
\begin{equation}
	a^k \equiv b^k mod m, \qquad k\in N_0
\end{equation}

Такие свойства достаточно упрощают всякие штуки, например:
\begin{equation}
	\sum_i {A_i b^i} mod m = \sum_i(A_i mod m)(b mod m)^i;
\end{equation}

Пример:
\begin{equation}
	17^{17} mod 13 \equiv
	4^{17} = (4^2)^8 \cdot 4 \equiv
	3^8\dot 4 \equiv
	(3^3)^2 \cdot 9 \cdot 4 \equiv
	36 \equiv 10 mod 13
\end{equation}

Еще пример:
\begin{equation}
 d|mL
 a \equiv b mod m => 
\end{equation}

Пусть числа $ m, n $ взаимнопростые, т.е. $ НОД(m,n) = 1 $ и $ a \equiv b mod m $ и $ a \equiv b mod n $ => $ a \equiv b mod mn $



(Тут рисуночек)
$ С_0  = [0] = \left\lbrace 0, +-m, +- 2m... \right\rbrace $
$ C_1 = [1] = \left\lbrace 1, +-m+1 , +-2m+1 \right\rbrace $
...
$ C_{m-1} = [m-1] = \left\lbrace m-1, +-m+(m-1) , +-2m+(m-1) \right\rbrace $
 - это полная система вычетов по модулю $ m $. Причем, классы можно складывать умножать и все такое.
 Для примера:
 $ C_i + C_j = C_{i+j} $
 например :
 $ (7+12)mod 5 = 19 mod 5 = 4 $
 $ 2+(-3) = (-1 mod 5) = 4 $

И умножать:
 $ c_i \cdot c_j = c_{ij}$
 
 
 Обратные по умножению:
 $ a^{-1}: a^{-1}\cdot a = 1 mod m $
 $ Z_m = \left\lbrace [0], [1], ... ,[m-1]\right\rbrace = \left\lbrace 0, 1, ... ,m-1\right\rbrace  $
 
 $ Z_5 = \left\lbrace 0, 1, 2,3,4\right\rbrace  $
 $ 1^{-1} = 1 $
 $ 2^{-1} = 3 $
 $ 3^{-1} = 2 $
 $ 4^{-1} = 4 $
 
 
 
 $ Z_6 = {0,1,2,3,4,5} $
 $ 1^-1 = 1 $
 $ 2^-1 = null $
 $ 3^-1 = null $
 $ 4^-1 = null $
 $ 5^-1 = 5 $
 
 
$ \exists a^{-1} mod m \Leftrightarrow (a,m)=1$  -взаимнопростые;


Алгоритм для нахождения обратного по модулю - расширенный алгоритм Эвклида.
$ (a,m)=1, \qquad m>a $ 
$ m=q_1 a + r_1 $
$ a = q_2 r_1 + r_2 $
$ r1 = q_3 r_2 + r_3 $
...
$ r_{n-3} = q_n r_{n-2} + {r_{n-1}}$
$ r_{n-2} = a_{n-2} + r_n$
$ r_{n-1} $ - нод
$ r_n = 0 $



$ 1 = um + va $ 
Возьмем по модулю $ m $ :
$ 1 \equiv va mod m $ -> $ v $ будет обратным к $ a $
Пример:
$ 23^{-1} mod 135 $
$ 135 = 5 \cdot 23 + 20 $
$ 23 = 1 \cdot 20 + 3 $
$ 20 = 6 \cdot 3 + 2 $
$ 3 = 1 \cdot 2 + 1 $
1 -> нод 23 и 135.
$ 1  = 3 - 1*2 = 3-1*(20-6*3) = -20 + 7*3 = -20 + 7*(23-20) = 7*23 - 8*20 = 7*23-8(135-5*23) = -8*135 + 47*23 $


\begin{equation}
	\begin{array}{cccccc}
		&  & q_1` & q_2 & ...  & q_n \\ 
		0 & 1 & p_1 & p_2 &  & p_n
	\end{array} 
\end{equation}
$ p_i = q_i p_{i-1} + p_{i-2} $

$ a^{-1}modm = (-1)^m p_n$

Пример: 
\begin{equation}
	\begin{array}{cccccc}
			& 	& 5 & 1 & 6  & 1 \\ 
		0 & 1 & 5 & 6 & 41 & 47
	\end{array} 
\end{equation}

Домашнее задание:

